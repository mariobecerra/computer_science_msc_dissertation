%!TEX root = ../msc_thesis.tex

\chapter{Conclusions}
\label{ch:conclusions}


In this work, three Bayesian acquisition functions were compared with two frequentist acquisition functions through a series of experiments in three different image datasets: the MNIST, the CIFAR10 and a cats and dogs dataset. An attempt was made to replicate the experiments performed with the MNIST dataset in \cite{Gal2016Active}, but failed in obtaining the same results in the comparison of approximate Bayesian and frequentist acquisition functions. However, the attempt was successful in achieving a better performance using a non-random acquisition function in contrast to a random one. The same experimental framework was used in the other two datasets. In the CIFAR10 dataset there seems to be a small performance improvement when using the non-random acquisition functions and in the cats and dogs dataset there is an even smaller performance improvement. As in the MNIST dataset, in these two other datasets there is no clear distinction between using an approximate Bayesian model against a frequentist one. The cause of this may be that the variational approximation to the posterior distribution is not very good. With an improvement of this approximation, the results may turn out to be better.

The use of a full posterior predictive distribution is of use in many different areas. One such area is autonomous vehicles, in which knowledge of the model's predictive uncertainty can help prevent accidents. If a model is uncertain about a prediction, then human assistance can be requested \cite{gal2016uncertainty, kendall2017uncertainties, michelmore2018evaluating}. Another area of opportunity lies in adversarial attacks, in which the use of Bayesian CNNs helps in providing less confident predictions for adversarial examples \cite{li2017dropout, rawat2017adversarial, smith2018understanding}. An additional area is the use of uncertainty for Reinforcement Learning to accelerate learning \cite{gal2016uncertainty}. One more general and very broad area of opportunity is in automatic classification systems using machine learning to make decisions, such as a post office automatically sorting letters according to a zip code, in which the model can decide to ask for the help of a human when the it is uncertain about certain predictions \cite{gal2016uncertainty}.

The code used to write this document is available in \url{https://github.com/mariobecerra/msc_thesis}.

And the code used for the experiments is available in \url{https://github.com/mariobecerra/Active_Learning_CNNs}.
